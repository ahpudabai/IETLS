
\documentclass[fleqn]{article}
\usepackage{amsmath}

\setlength{\mathindent}{0pt} % 设置公式与页面左边的距离

% Language setting
% Replace `english' with e.g. `spanish' to change the document language
\usepackage[english]{babel}
\usepackage{CJKutf8}

% Set page size and margins
% Replace `letterpaper' with `a4paper' for UK/EU standard size
\usepackage[letterpaper,top=2cm,bottom=2cm,left=3cm,right=3cm,marginparwidth=1.75cm]{geometry}
\usepackage[fontset=ubuntu]{ctex}
% Useful packages
\usepackage{amsmath}
\usepackage{graphicx}
\usepackage{array}
\usepackage[colorlinks=true, allcolors=blue]{hyperref}
\usepackage{booktabs} % Pretty tables
\title{English Class Notes Of The Sixth Group}
\author{Gerry Zhou}

\begin{document}
\maketitle

例1 a为整数, 则 a = 3  C

(1) a> 2
(2) a< 4

下推上


直线 y = ax+b 与抛物线 y= x^{ 2 } 有两个交点

(1) a^{ 2 } > 4b
(2) b>0

x^{2} - ax - b = 0


若x,y为整数,且xy=12, 则 x的值确定 

(1) \frac{x}{6} 为整数
(2) \frac{y}{2} 为整数


实数R 有理数Q(4/2  1/3  无限循环小数) 无理数(无限不循环小数: pi e  = 2.71828  limit (1+x)1/x = e) 标准能不能写成分数 \frac{pi}{2}   不是分数
2 = 1.414 3 = 1.732  5 = 2.236  log32  log23  

正整数  质数 和 合数
质数  2, 3, 5, 7, 11, 13, 17, 19
4, 6, 8, 9, 19

奇数加奇数  
奇数x奇数 奇数

加减 是奇数


abc = 5(a+b+C)

定性和定量

(\frac{1}{2} + \dots + \frac{1}{2007}())


对于同一个除数m, 两个数据


余数 差,积  和 同余
16x41 除以 7的余数为


\end{document}